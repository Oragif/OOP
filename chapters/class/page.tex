\chapter{Classes}
Chapter begins: \jfun[225]

\section{What is a Java Class}
Java Classes has a set of properties and functionalities. Non-static classes can be instantiated. Classes can also work with generic types and inlusion as mentioned in \hyperref[sec:parametric_polymorphism]{polymorphism} and \hyperref[sec:inclusion_polymorphism]{inclusion respectivly}. Include the following:
\begin{itemize}
    \item Fields
    \item Methods
    \item Constructors
    \item Static initializers
    \item Instance initializers
\end{itemize}
Basis for creating a class:
\begin{lstlisting}[language=Java]
    // <T> is only for generic classes
    [modifiers] class class-name <T> {
        // Body of the class goes here
    }
    // Example
    public abstact class Human<T> {
        // An empty body for now
    }
\end{lstlisting}
Static fields in a class are class fields, whereas none static fields are instance variables.

\subsection*{Imports}
Import statements that uses the wild(*) modifier are on demand imports, meaning all available classes and functions can be used on demand:
\begin{lstlisting}[language=Java]
    import com.java.string.*
\end{lstlisting}
The Java compiler must resolve the simple name A to its fully qualified name during the compilation process. It searches for a type referenced in a program in the following order:
\begin{itemize}
    \item The current compilation unit
    \item Single-type import declarations
    \item Types declared in the same package
    \item Import-on-demand declarations
\end{itemize}
